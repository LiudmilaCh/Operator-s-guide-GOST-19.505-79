
%\documentclass[a4paper,12pt]{article}
\documentclass[11pt]{article}
%\usepackage[a4paper, left=25mm, right=10mm, top=20mm, bottom=20mm]{geometry}
\usepackage{graphicx} % Required for inserting images
\usepackage{times}
\usepackage{titlepic}
\usepackage{tcolorbox}
\usepackage{graphicx}
\usepackage{amsmath}
\usepackage{multirow}
\usepackage{tikz}
\usepackage{array}
\usepackage{caption}
\usepackage[russian,english]{babel}
\usepackage{hyphenat}
\usepackage[T2A]{fontenc}
\usepackage[utf8]{inputenc}
\usepackage[russian]{babel}
\usepackage{tabularx}
\usepackage{booktabs} % for better looking tables



\usepackage[
   left=2cm, top=0.5cm, bottom=0.5cm, right=0.5cm,
    a4paper, portrait]{geometry} 
    
\thispagestyle{empty}
\begin{document}
        
\selectlanguage{russian}    

\begin{titlepage}
  
    \nointerlineskip\noindent%
\vspace*{-\topskip}%
%рамка
\begin{tikzpicture}
    \draw (0,0) rectangle (18.5cm,28.7cm) ;    
\end{tikzpicture}


% табличка сбоку
\vspace{85mm}
\begin{table} [ ht -12mm]

      \begin{tabular}{|>{}p{5mm}|>{}p{7mm}|}
        \hline
        \parbox[c][35mm][b]{5mm}{\rotatebox[origin=c]{90}{Подп.и дата}} &  \parbox[c][35mm][c]{7mm}{}\\ \hline
        \parbox[c][25mm][b]{5mm}{\rotatebox[origin=c]{90}{Инв.№дубл}} &  \parbox[c][25mm][c]{7mm}{}\\ \hline
        \parbox[c][25mm][b]{5mm}{\rotatebox[origin=c]{90}{Взам.инв.№}} &  \parbox[c][25mm][c]{7mm}{}\\ \hline
        \parbox[c][35mm][b]{5mm}{\rotatebox[origin=c]{90}{Подп.и дата}} &  \parbox[c][35mm][c]{7mm}{}\\ \hline
        \parbox[c][25mm][b]{5mm}{\rotatebox[origin=c]{90}{Инв.№подп}} &  \parbox[c][25mm][c]{7mm}{}\\ \hline
    \end{tabular}
\end{table} 
\end{titlepage}

\newpage
\section{Аннотация}

\newpage
%содержание
\tableofcontents

\newpage
\section{Назначение программы}
\input{section1.txt} % непосредственно подтягивается текст для этого раздела, все форматирование автоматически применяется

\newpage
\section{Условия выполнения программы}
\input{section2.txt} % непосредственно подтягивается текст для этого раздела, все форматирование автоматически применяется

\newpage
\section{Выполнение программы}
\input{section3.txt} % непосредственно подтягивается текст для этого раздела, все форматирование автоматически применяется

\newpage
\section{Сообщения оператору}
\input{section4.txt} % непосредственно подтягивается текст для этого раздела, все форматирование автоматически применяется

\newpage
%\section{ЛИСТ РЕГИСТРАЦИИ ИЗМЕНЕНИЙ}
\begin{table}[ht]
    \centering
    \begin{tabularx}{\textwidth}{|X|X|X|X|X|X|X|X|X|X|}
        \hline
        \multicolumn{10}{|c|}{\textbf{ЛИСТ РЕГИСТРАЦИИ ИЗМЕНЕНИЙ}} & \\
        \hline
        \multicolumn{5}{|c|}{\textbf{Номера листов (страниц)}} & \textbf{Всего листов (страниц) в докум} & \textbf{№ документа} & \textbf{Входящий № сопроводительного документа и дата} & \textbf{Подп.} & \textbf{Дата} \\
        \hline
        \textbf{Изм} & \textbf{изменен-ных} & \textbf{заменен-ных} & \textbf{новых} & \textbf{анулиро
ванных
} &  &  &  &  &  \\
        \hline
      &&&&&&&&& \\
      \hline
     &&&&&&&&& \\
      \hline
      &&&&&&&&& \\
      \hline
      &&&&&&&&& \\
      \hline
      &&&&&&&&&  \\
      \hline
 %     &&&&&&&&& \\
 %     \hline
 %     &&&&&&&&& \\
 %     \hline
 %     &&&&&&&&& \\
 %     \hline
 %     
    \end{tabularx}

\end{table}

\end{document}


